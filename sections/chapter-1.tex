\documentclass[../main.tex]{subfiles}

\begin{document}
\subsection{Q1}
\subsubsection*{(a)}
Probability distribution of x and y, use 
\begin{equation}
	P(x) = \sum_i \frac{p(g_i(x))}{| f'(g_i(x)) |}
\end{equation}
We have $x=f(\phi) = a\cos\phi$, $g_1(x) = \arccos(\frac{x}{a})$, 
$g_2(x) =2\pi- \arccos(\frac{x}{a})$, $|f'(\phi)|=|a\sin\phi|= |a\sqrt{1-\cos^2\phi}|$ 
So:
\begin{equation}
	P(x) = \frac{1}{\pi\sqrt{a^2-x^2}}
\end{equation}
And the same for $P_x(x) = P_y(y)$ from symmetry.
\subsubsection*{(b)}
For the join probability distribution, we can use 
\begin{equation}
	p(x, y) = \mathcal{P}(x|y)p(y)
\end{equation}
Where $\mathcal{P}(x|y)=\frac{1}{2}\delta(x^2+y^2-a^2)$, so:
\begin{equation}
	p(x, y) = \frac{\delta(x^2+y^2-a^2)}{2\pi\sqrt{(a^2-y^2)}}
\end{equation}
\subsubsection*{(c)}
The covariance of x and y is: $\left<\Delta x \Delta y\right>=\left<(x-\left<x\right>)(y-\left<y\right>)\right>=\left<xy\right>-\left<x\right>\left<y\right>$ Which is zero because the probability distribution $p(x)$ is even.
\\
The distributions are statistically dependent since $p(x, y)\neq p(x)p(y)$.
\subsection{Q2}
The characteristic function of the random variable $x$ is:

\begin{equation}
	C(\xi) = 
	\begin{cases}
		\frac{b+\xi}{b-a} & -b<\xi\leq-a \\
		1 & a-<\xi\leq a\\
		\frac{b-\xi}{b-a} & a<\xi\leq b \\
		0 & \text{Otherwise}
	\end{cases}
\end{equation}
Can get the probability distribution as the fourier transform of this:
\begin{equation}
	p(x) = \frac{1}{2\pi}\int e^{-i\xi x}C(\xi)d\xi
\end{equation}
Since $C(\xi)$ is even, only the even part of the exponential contributes, so we need to perform the integral:
\begin{equation}
	p(x) = \frac{1}{2\pi}\int_{-b}^b C(\xi)\cos(x\xi)d\xi
\end{equation}
Perform the integral over the 3 parts, giving:
\begin{equation}
	p(x) = \frac{\cos(ax)-\cos(bx)}{\pi(b-a)x^2}
\end{equation}
\subsection{Q3}
Probability of $m$ failures and $n$ successes, with last trial ending in success. Probability of a success is $\alpha$.
\begin{equation}
	p(m, n) = \left(\begin{matrix}m+n-1\\m\end{matrix}\right)(1-\alpha)^m\alpha^n
\end{equation}
The r'th factorial moment of $m$ for fixed $n$ is
\begin{align*}
	\left<m^{(r)}\right>=\frac{d^r}{d\xi^r}\left<(1+\xi)^m\right>\Big |_{\xi=0} \\
			    =\frac{d^r}{d\xi^r}\sum_{m=0}^\infty (1+\xi)^m \left(\begin{matrix}m+n-1\\m\end{matrix}\right)(1-\alpha)^m\alpha^n\Big|_{\xi=0} \\
			    =\alpha^n\frac{d^r}{d\xi^r}\sum_{m=0}^\infty\left(\begin{matrix}-n\\m\end{matrix}\right)(\alpha-1+\alpha\xi-\xi)^m\Big|_{\xi=0} \\
			    =\alpha^n \frac{d^r}{d\xi^r}(\alpha+\alpha\xi-\xi)^{-n}\Big|_{\xi=0} \\
			    =\alpha^n (1-\alpha)^r\alpha^{1-n-r}\frac{(n+r-1)!}{(n-1)!} \\
			    =\alpha^{1-r}(1-\alpha)^r\frac{(n+r-1)!}{(n-1)!}
\end{align*}
Where from the 3rd to the 4th line the binomial expansion is used.
\subsection{Q4}
Let $x$ be the position of the division and $a$ and $b$ be the lengths of the first and second parts of the divided line. The probability distribution of $x$ is $p(x)=1$.
\subsubsection*{(a)}
The average length of $a$ is:
\begin{align*}
	\left<a\right> = \int_0^1a p(a)da = \left[\frac{a^2}{2}\right]^1_0 = \frac{1}{2}
\end{align*}
From symmetry the length of $b$ is the same.
\subsubsection*{(b)}
The average ratio of shorter to longer is given by:
\begin{align*}
	\left<\text{average ratio}\right> = \int_0^{\frac{1}{2}} \frac{a}{1-a}da + \int_{\frac{1}{2}}^1\frac{1-a}{a}da\\
	=2\int_{\frac{1}{2}}^1\frac{1-a}{a}da\\
	=2\int_{\frac{1}{2}}^1\frac{1}{a}da - 2\int_{\frac{1}{2}}^1da \\
	=2\left[\ln\left(a\right)\right]_{\frac{1}{2}}^1 - 1 \\
	=-2\ln\left(\frac{1}{2}\right)-1\\
	=\ln(4)-1
\end{align*}
\subsubsection*{(c)}
For the covariance matrix $\mu_{ij} = \left(\begin{matrix}\left<(\Delta a)^2\right> & \left<\Delta a\Delta b\right> \\ \left<\Delta b\Delta a\right> & \left<(\Delta b)^2\right>\end{matrix}\right)$, we only need to calculate $\left<(\Delta a)^2\right>$ and $\left<\Delta a\Delta b\right>$ because of the symmetry between $a$ and $b$. 
\begin{align*}
	\left<(\Delta a)^2\right> = \left<a^2\right> - \left<a\right>^2 = \int_0^1a^2da - \frac{1}{4} = \frac{1}{12}
\end{align*}
For the correlation between the lengths, we need $p(a, b)$, which is just:
\begin{equation}
	p(a, b) = \delta(a+b-1)
\end{equation}
\begin{align*}
	\left<\Delta a\Delta b\right> = \int_0^1ada\int_0^1b\delta(a+b-1)db - \left<a\right>\left<b\right> \\
	= \int_0^1a (1-a)da - \frac{1}{4} \\
	= \frac{1}{2}-\frac{1}{3}-\frac{1}{4} = -\frac{1}{12}
\end{align*}
And the covariance matrix is:
\begin{equation}
	\mu = \frac{1}{12}\left(\begin{matrix}1 & -1 \\ -1 & 1\end{matrix}\right)
\end{equation}
\subsection{Q5}
The $\chi^2$ value is given by:
\begin{equation}
	\chi^2 = \sum_i \left(\frac{X_i-\mu}{\sigma_i}\right)^2
\end{equation}
Where $X_i$ are the values measured, $\mu$ is the average, or expected value and $\sigma_i$ are the errors.
With the null hypothesis being the student gathered the values ``legally'', we say $\mu=981\text{cm/s$^2$}$ and the error is $1\%$ of this, so $\sigma=9.81\text{cm/s$^2$}$. The $\chi^2$ value is 0.05. The test has 4 degrees of freedom, and so a value of $\chi^2=0.05$ or higher has a probability $p>0.995$. The student probably didn't cheat.
\subsection{Q6}
From n people, the probability of at least two people sharing a birthday is $p(n)=1-\bar{p(n)}$ where $\bar{p(n)}$ is the probability of no people sharing a birthday. $\bar{p(n)}$ is the number of ways to choose $n$ separate days from $365$, $\left(\begin{matrix}365 \\ n\end{matrix}\right)$ divided by the total number of ways those $n$ days could be distributed to different people, $\frac{365^n}{n!}$. We have:
\begin{align*}
	p(n) = 1-\frac{n!\left(\begin{matrix}365\\n\end{matrix}\right)}{365^n}\\
	= 1-\frac{365!}{365^n(365-n)!}\\
	=1-\left(1-\frac{0}{365}\right)\left(1-\frac{1}{365}\right)\mathellipsis\left(1-\frac{n-1}{365}\right)
\end{align*}
We can use the first order expansion $e^{-x}\approx 1-x$ to replace the terms in the product with exponentials:
\begin{align}
	p(n)\approx 1-e^{-\frac{0}{365}}e^{-\frac{1}{365}}\mathellipsis e^{-\frac{n-1}{365}}\\
	=1-\exp\left(-\frac{1}{365}\sum_{i=1}^n(i-1)\right)\\
	=1-\exp\left(-\frac{1}{365}\frac{n^2-n}{2}\right)\\
	=1-\exp\left(\frac{n-n^2}{730}\right)
\end{align}
Which was to be shown.
The number of people required for the probability of people sharing a birthday to be $>\frac{1}{2}$ is given by solving:
\begin{align}
	p(n) > \frac{1}{2}\\
	1-\exp\left(\frac{n-n^2}{730}\right)>\frac{1}{2}\\
	\exp\left(\frac{n-n^2}{730}\right)<\frac{1}{2}\\
	n^2-n-730\ln(2)>0
\end{align}
The smallest integer that satisfies this is $n=23$.
\subsection{Q7}
We want to find the probability distribution of $q$, $\mathcal{P}(q|N)$, the probability for the system to be in state $A$ given that $A$ was found $N$ times from the $2N$ trials.
This can be found using Bayes' theorem:
\begin{align}
	\mathcal{P}(q|N) = \frac{\mathcal{P}'(N|q)p(q)}{p(N)} = \frac{\mathcal{P}'(N|q)p(q)}{\int_0^1\mathcal{P}'(N|q)p(q)dq}
\end{align}
Since we have no prior knowledge about $q$, we have $p(q)=1$. $\mathcal{P}'(N|q)$ is given by the binomial distribution: 
\begin{equation}
	\mathcal{P}'(N|q)=\left(\begin{matrix}2N\\N\end{matrix}\right)q^N(1-q)^N
\end{equation}
The integral $\int_0^1q^N(1-q)^Ndq$ can be performed by repeated integration by parts to get:
\begin{equation}
	\int_0^1q^N(1-q)^Ndq = \left((2N+1)\left(\begin{matrix}2N\\N\end{matrix}\right)\right)^{-1}
\end{equation}
And we have:
\begin{equation}
	\mathcal{P}'(q|N) = (2N+1)\left(\begin{matrix}2N\\N\end{matrix}\right)q^N(1-q)^N
\end{equation}
Which is plotted for different values of $N$ in figure \ref{fig:bayes_prob_dists}.
\begin{figure}
\centering
\subfloat{\includesvg[width=0.8\textwidth]{./bayes_graphs.svg}
}\qquad
\caption{Probability distributions of $q$ for different values of $N$}
\label{fig:bayes_prob_dists}
\end{figure}
\subsection{Q8}
Let $X_n$ be the probability of falling off the cliff when $n$ steps away, $\alpha$ be the probability that the lad steps toward the cliff-edge. We know $X_0$ = 1, and that:
\begin{equation}
	X_n=\alpha X_{n-1} + (1-\alpha) X_{n+1}
\end{equation}
If we look at $X_1$ specifically:
\begin{equation}
		X_1=\alpha+(1-\alpha)X_2
\end{equation}
And we know that the probability of falling from each point is given by $X_n=X_1^{n}$, so we have an algebraic equation for $X_1$:
\begin{align}
	X_1=\alpha+(1-\alpha)X_1^2
\end{align}
Which gives:
\begin{equation}
	X_1=1,\frac{\alpha}{1-\alpha}
\end{equation}
And the probability of falling off at $n$ steps is:
\begin{equation}
	X_n = \left(\frac{\alpha}{1-\alpha}\right)^n
\end{equation}
\subsection{Q9}
We have probability distribution:
\begin{equation}
	p(\mathbf{x}) = \frac{1}{(2\pi)^{\frac{N}{2}}\det{\mu}^{\frac{1}{2}}}\exp{\left(-\frac{1}{2}\mathbf{x}^\dag\mu^{-1}\mathbf{x}\right)}
\end{equation}
We have a transformation to $\mathbf{y}=U\mathbf{x}$ with $U$ being unitary. The transformation leaves $\det{\tilde\mu}=\det{U\mu U^\dag}=\det{U}\det{\mu}\det{U^\dag}=\det{\mu}$ the same, and we have $\mathbf{x^\dag}\mu^{-1}\mathbf{x}=\mathbf{y^\dag}\tilde\mu^{-1}\mathbf{y}$
Calculating:
\begin{align}
(1-\rho^2)\tilde\mu = \left(\begin{matrix}\cos\theta & -\sin\theta \\ \sin\theta & \cos\theta\end{matrix}\right) 
\left(\begin{matrix}\frac{1}{\sigma_x^2} & -\frac{1}{\sigma_x\sigma_y} \\ - \frac{1}{\sigma_x\sigma_y} & \frac{1}{\sigma_y^2}\end{matrix} \right)
\left( \begin{matrix}\cos\theta & \sin\theta \\ -\sin\theta & \cos\theta\end{matrix}\right)
\end{align}
Which has diagonal form if $\sin\theta\cos\theta=0$, so $\theta=\frac{n\pi}{2}, n\in\mathcal \mathbb{N}$
\end{document}
